\documentclass[letterpaper,titlepage]{article}
\usepackage[spanish]{babel}
\usepackage[utf8]{inputenc}
\usepackage{url}
\usepackage{graphicx}

\title{Marcado de la Información sobre Parlamentarios \\ con RDFa}
\author{Biblioteca del Congreso Nacional}

\begin{document}

\maketitle

\thispagestyle{empty}

\abstract

Este informe propone soluciones para el marcado con RDFa de la
información asociada a parlamentarios, publicada por la Biblioteca del
Congreso Nacional (BCN) a través de la Web. Esta propuesta se
componen de tres etapas. Primero, se define un esquema para marcar
con RDFa. Luego, en base al esquema propuesto, se presentan
vocabularios existentes, o definen nuevos cuando no se encuentran
vocabularios apropiados, para usarlos en el marcado de los recursos
definidos en el esquema. Finalmente, se genera una serie de
recomendaciones para implementar el marcado. \medskip

Este informe surge en el contexto de una asesoría solicitada por el
Área de Arquitectura de la Información de la BCN. El informe es
redactado por un asesor en el área, en base a las
ideas y supervisión del equipo del Área de Arquitectura de la
Información.

\newpage

\thispagestyle{empty}

\null

\vfill

\begin{description}
\item[Derechos] \hfill \\
  Este informe es publicado por la BCN
  bajo licencia Creative Commons Atribución-Licenciar Igual 3.0. \\
  \url{http://creativecommons.org/licenses/by-sa/3.0/deed.es_CL}
\item[Estatus] \hfill \\
  Este documento es un borrador en progreso.
  Las fuentes de este informe pueden descargarse desde: \\
  \url{http://github.com/danielhz/chile_schema}.
\item[Editores] \hfill \\
  {\sf Daniel Hernández} \url{<daniel@degu.cl>} \\
  {Consultor en publicación de datos en la Web.} \\
  {\sf Roxana Donoso} \\
  {Jefe Área Arquitectura de Información, BCN.} \\
  {\sf Felipe Almazan} \\
  {Arquitecto de Información, BCN.} \\
  {\sf Claudio Gutierrez} \\
  {Académico, Departamento de Ciencias de la Computación, Universidad de Chile}
\end{description}

\newpage

\tableofcontents
%\listoffigures

\newpage

\section{Introducción}

¡Este es un borrador! Se irá completando a medida que se vaya
desarrollando el proyecto.

\section{Programa}

Esta sección describe el programa para el desarrollo de este
informe. Eventualmente esta sección será removida al término de la
asesoría.

\begin{enumerate}
\item {\em Identificación de datos a modelar}. En esta etapa se
  identificará a los participantes en la información que se va a
  modelar. Esto implica identificar candidatos a clases y candidatos a
  predicados. Esto es un proceso que podría abarcar prácticamente todo
  el desarrollo, puesto que suele ocurrir que se descubran elementos
  que no habían sido considerados. No obstante, la identificación de
  datos a modelar espera hacerse en las primeras dos
  reuniones\footnote{Ya llevamos una reunión}. La metodología a seguir
  para identificar los datos a modelar va a ser la conversación con
  los integrantes del equipo de la BCN, la presentación de los
  conceptos identificados por parte del equipo asesor en RDF y la
  validación de estos mismos por parte del equipo de la BCN.
\item {\em Exploración de vocabularios existentes}. Paralelamente a la
  identificación de datos a modelar, se irán identificando
  vocabularios existentes que podrían tener relación con los elementos
  identificados. Con estos antecedentes se podrá tomar la
  decisión de reutilizar los vocabularios existentes o de utilizar
  nuevos vocabularios.
\item {\em Taller de modelado}. Se debiera realizar un pequeño taller
  de modelado usando triples, que ayude al equipo de la BCN a
  comprender mejor lo que se está haciendo. Probablemente esto debiera
  hacerse en la tercera reunión, luego de haber identificado gran
  parte de los datos a modelar y luego que se desarrollará un poco más
  la intuición sobre el modelado con triples.
\item {\em Identificadores para objetos}. Para cada tipo de objeto se
  debe proponer un formato para expresar sus identificadores como
  URIs. Esto podría hacerse en la cuarta sesión.
\item {\em Identificadores para predicados y clases}. Paralelo a la
  identificación de datos se irán validando los nombres para los
  predicados y las clases, lo que podrían quedar claros hacia la
  quinta sesión.
\item {\em Taller de RDFa}. Luego de haber modelado la
  información, se debiera dejar una sesión para hacer un taller de uso
  de RDFa, donde se indicará como hacer el marcado. Probablemente está
  será la quinta sesión.
\item {\em Marcado}. Se debe estudiar las posibilidades de marcado en
  las herramientas de publicación usadas por la BCN, para poder ver
  hasta que nivel las herramientas podrían facilitar el marcado,
  hacerse un trabajo manual o implementar una herramienta que lo
  facilite. Según las herramientas usadas y la estructura de los
  documentos XHTML publicados, se verá como introducir el marcado en
  ellos utilizando RDFa. Esta tarea podría ser en la sexta sesión.
\end{enumerate}

\section{Preguntas}

Estas son algunas preguntas para responder, que se irán quitando a
medida que sean respondidas:

\begin{enumerate}
\item ¿Los períodos parlamentarios pueden ser de 4 u 8 años o todos
  los períodos son de 4 años y los senadores son elegidos por dos
  períodos?
\end{enumerate}

\section{Técnica de modelado}
\label{tecnica-de-modelado}

El modelamiento se basa en RDF, que es en la actualidad el marco
conceptual que sustenta iniciativas como la Web Semántica y el
movimiento Linked Data.

En RDF la unidad de los datos es el triple, es decir, una expresión
compuesta de tres partes: sujeto, predicado y objeto. Cada una de
estas partes es representada por una URI o un nodo blanco. Además, los
objetos pueden tomar valores literales. Las URI, corresponden a los
identificadores en la web (Uniform Resource Identifier) que hoy en
día, para fomentar la adopción en otros idiomas, están siendo
reemplazados por las IRIs (Internationalized Resource Identifier).

Ante la necesidad de identificar triples para poder hacer afirmaciones
sobre ellos sin recurrir a técnicas de reificación, se ha extendido el
uso de grafos, que, permiten asociar un conjunto de triples con una
IRI que representa el nombre del grafo. En la práctica los grafos
pueden verse como una extensión al modelo de triples mediante una
cuarta componente, por lo que las tuplas $(g,s,p,o)$ son también
llamadas quads.

El lenguaje de consulta para RDF, SPARQL, y varias de sus
implementaciones existentes incorporan hoy el concepto de
grafos. También existen varias extensiones, aun no del todo
estandarizadas que utilizan el concepto de grafos, como TriG, TriX y
N-Quads. No obstante lo anterior, aun no existe una manera de publicar
quads en RDFa, por lo que en los modelos presentados no se hará uso
del concepto de grafos.

Los estándares están sujetos a un constante cambio, no obstante la
información publicada debe persistir. Por ello en este informe los
esquemas son presentados en dos faces. En la primera se utiliza el
español para definir los conceptos y sus relaciones, sin entrar a
restringir demasiado el esquemas. En la segunda fase el esquema
desarrollado es mapeado a vocabularios existentes y usando los
estándares de moda. De este modo, se gana independencia de estándares
y vocabularios y se genera una base conceptual mínima para que la
información sea mapeada a nuevos estándares.

La no restricción del esquema con relaciones, con cardinalidades,
jerarquías de clases y definición de clases disjuntas, entre otros
elementos de lenguajes de representación del conocimiento como OWL,
sigue las siguientes razones: $a$) quienes son responsables de publicar
los datos conocen de ellos lo suficiente como para ingresar los datos
de la forma adecuada, y $b$) modelar utilizando lenguajes como OWL
resulta, intentando alcanzar la perfección en la representación del
conocimiento, resulta demasiado trabajoso para los beneficios que
ofrece, pudiendo incluso, enlentecer el proceso de publicación.

En vez de usar lenguajes de representación del conocimiento, en este
informe se opta por utilizar lenguajes de modelado conceptual (como
por ejemplo Entidad-Relación). En general los lenguajes de modelado
conceptual poseen, tanto una semántica formal, como una representación
gráfica que resulta especialmente útil para la discusión sobre los
modelos durante el proceso de modelado por un equipo formado por
una parte experta en el lenguaje y otra experta en los conceptos a
modelar. La semántica del lenguaje de modelado a utilizar en este
informe se define como sigue: \medskip

\noindent {\bf Definición:} Dado un modelo $\mathcal{M}$, definido
como un conjunto de triples $(S,p,O)$, donde $S$ y $O$ representan
clases y $p$ representa a un predicado, decimos que: un conjunto de
triples $T$ satisface el modelo $\mathcal{M}$ si, para todo triple
$(s,p,o)$ en $T$, existe un triple $(S,p,O)$ en $\mathcal{M}$ tal que
$(s,o) \in S \times O$. \medskip

Dada la naturaleza positiva del lenguaje RDF, haciendo uso del
predicado \url{rdf:type} podría responderse positivamente a la
pregunta sobre si un conjunto de triples satisface un modelo. No
obstante, responder a la pregunta sobre si no lo satisface requiere
suponer negar el predicado \url{rdf:type}, lo que no es posible en
con la semántica de RDF. Bajo ciertas reglas aportadas por lenguajes
como OWL detectar datos que no satisfagan el modelo podría ser
posible, no obstante, tal como se indico en la razón $b$), se cree que
no es necesario, a este nivel, poner tanto énfasis en la validación
automática de los datos publicados.

El lenguaje de modelado conceptual propuesto posee una representación
gráfica que es explicada en la sección \ref{diagramas}, donde también
se presentan diagramas que facilitan la comprensión de los modelos
propuestos.

\section{Identificación de datos a modelar}

Tal como se propuso en la sección \ref{tecnica-de-modelado}, las
componentes de los triples del esquema son o clases o predicados. No
obstante en la práctica se diferencian las clases en clases
propiamente tales, que agrupan recursos identificadas por una IRI o un
nodo blanco, y tipos de datos, que definen conjuntos de literales,
tales como números, booleanos, texto, etc. Para facilitar la
identificación, se usarán los prefijos \#, @ y \$ para distinguir
entre clases, predicados y tipos de datos, respectivamente. Además,
por convención, se usará UperCamelCase para los nombres de clases y
lowerCamelCase para los nombres de predicados y tipos de datos. Por
ejemplo, el esquema considera la clase \#PersonaNatural, el predicado
@nombre y el tipo de datos \$texto.

Esta sección se divide en dos subsecciones, \ref{clases} y
\ref{predicados}, que describen las clases y los predicados usados en
el esquema, respectivamente.

\subsection{Clases}
\label{clases}

En esta sección se describirán las clases del esquema
propuesto, indicando para cada una de ellas las propiedades entrantes
y las propiedades salientes. Cada propiedad entrante de una clase $A$,
corresponde a un par $(p,O)$ tal que $(A,p,O)$ está en el modelo y
cada propiedad saliente corresponde a un par $(p,S)$ tal que $(S,p,A)$
está en el modelo.

\input{classes.tex}

\subsection{Predicados}
\label{predicados}

\input{predicates.tex}

\section{Potencial de los datos}

La información publicada podría ser útil para organizaciones de la
sociedad civil, tales como la fundación Ciudadano Inteligente o Ojo
con el Parlamento, que la presentan de manera que pueda influir en las
votaciones, mediante ciudadanos mejor informados. Información sobre
las pertenencias a partidos o a bancadas podría ayudar a la población
a hacerse un mapa más detallado del congreso en relación a los
parlamentarios y a las agrupaciones de estos, como los partidos. Estas
relaciones serían aún más finas si se vinculan con otras fuentes de
información como son las votaciones. Ello permitiría definir espacios
muy similares a los que se definen en los estudios de componentes
principales o a la definición de grafos donde los arcos pueden ser
votaciones similares. Acotando estos estudios por materias (medio
ambiente, economía, educación,...) podría obtenerse otras
visualizaciones interesantes.

\newpage
\appendix

\section{Diagramas de esquemas}
\label{diagramas}

\subsection{Personas naturales}
\begin{center}
  \includegraphics[width=\textwidth]{diagrams/dia-persona.pdf}
\end{center}

\subsection{Diputados y senadores}
\begin{center}
  \includegraphics[width=\textwidth]{diagrams/dia-diputados-senadores.pdf}
\end{center}

\subsection{Subdivisiones territoriales}
\begin{center}
  \includegraphics[width=\textwidth]{diagrams/dia-lugares.pdf}
\end{center}

\subsection{Procesos eleccionarios}
\begin{center}
  \includegraphics[width=\textwidth]{diagrams/dia-elecciones.pdf}
\end{center}

\end{document}