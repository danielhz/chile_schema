\subsubsection{@apellidoMaterno}

El predicado @apellidoMaterno sólo es usado para describir personas
naturales, es decir, es un atributo propio de la clase
\#PersonaNatural.

\begin{description}
  \input{node_predicates/predicate_apellidoMaterno.tex}
\item[{\sf Vocabularios relacionados:}]
  Para representar apellidos, el vocabulario FOAF propone dos
  predicados:
  \url{foaf:familyName} y \url{foaf:lastName}. En la especificación de
  FOAF se recomienda usar \url{foaf:familyName} por sobre
  \url{foaf:lastName}, porque el segundo posee un ámbito cultural más
  restringido. FOAF no provee una forma de distinguir entre los
  apellidos paterno y materno, lo que significa que si se quiere hacer
  una distinción, se debería agregar un predicado específico para ello
  al vocabulario. Además, esto abre dos posibilidades de marcado:
  publicar los atributos marcándolos con las dos propiedades o heredar
  el predicado @apellidoMaterno desde \url{foaf:familyName}.
\item[{\sf Predicados relacionados}]
  El predicado @apellidoMaterno es particularización del
  predicado @name y puede ser considerado un hermano del predicado
  @apellidoPaterno.
\end{description}
