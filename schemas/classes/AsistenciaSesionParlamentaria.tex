\subsubsection{\#AsistenciaSesiónParlamentaria}

Los registros de asistencia, tal como son presentados en los
sitios del Senado \footnote{\url{http://www.senado.cl}} y la Cámara de
Diputados \footnote{\url{http://www.camara.cl}} no son exactamente
iguales. En el sitio del Senado sólo se listan los nombres de los
asistentes, mientras que en el sitio de la Cámara de Diputados se
agrega una observación junto a los nombres de algunos
inasistentes. Las observaciones son típicamente una palabra como
``aviso'' o ``licencia''. No todas las inasistencias van acompañadas
de una observación.

Ante esta diferencia entre las formas de presentación de la
información, se decidió definir el vocabulario siguiendo el esquema
presentado por la Cámara de Diputados, puesto que es más completo y
permite también cubrir la publicación que de asistencias a las
sesiones del Senado.

\begin{description}
  \subsubsection{\#AsistenciaSesiónParlamentaria}

Los registros de asistencia, tal como son presentados en los
sitios del Senado \footnote{\url{http://www.senado.cl}} y la Cámara de
Diputados \footnote{\url{http://www.camara.cl}} no son exactamente
iguales. En el sitio del Senado sólo se listan los nombres de los
asistentes, mientras que en el sitio de la Cámara de Diputados se
agrega una observación junto a los nombres de algunos
inasistentes. Las observaciones son típicamente una palabra como
``aviso'' o ``licencia''. No todas las inasistencias van acompañadas
de una observación.

Ante esta diferencia entre las formas de presentación de la
información, se decidió definir el vocabulario siguiendo el esquema
presentado por la Cámara de Diputados, puesto que es más completo y
permite también cubrir la publicación que de asistencias a las
sesiones del Senado.

\begin{description}
  \subsubsection{\#AsistenciaSesiónParlamentaria}

Los registros de asistencia, tal como son presentados en los
sitios del Senado \footnote{\url{http://www.senado.cl}} y la Cámara de
Diputados \footnote{\url{http://www.camara.cl}} no son exactamente
iguales. En el sitio del Senado sólo se listan los nombres de los
asistentes, mientras que en el sitio de la Cámara de Diputados se
agrega una observación junto a los nombres de algunos
inasistentes. Las observaciones son típicamente una palabra como
``aviso'' o ``licencia''. No todas las inasistencias van acompañadas
de una observación.

Ante esta diferencia entre las formas de presentación de la
información, se decidió definir el vocabulario siguiendo el esquema
presentado por la Cámara de Diputados, puesto que es más completo y
permite también cubrir la publicación que de asistencias a las
sesiones del Senado.

\begin{description}
  \subsubsection{\#AsistenciaSesiónParlamentaria}

Los registros de asistencia, tal como son presentados en los
sitios del Senado \footnote{\url{http://www.senado.cl}} y la Cámara de
Diputados \footnote{\url{http://www.camara.cl}} no son exactamente
iguales. En el sitio del Senado sólo se listan los nombres de los
asistentes, mientras que en el sitio de la Cámara de Diputados se
agrega una observación junto a los nombres de algunos
inasistentes. Las observaciones son típicamente una palabra como
``aviso'' o ``licencia''. No todas las inasistencias van acompañadas
de una observación.

Ante esta diferencia entre las formas de presentación de la
información, se decidió definir el vocabulario siguiendo el esquema
presentado por la Cámara de Diputados, puesto que es más completo y
permite también cubrir la publicación que de asistencias a las
sesiones del Senado.

\begin{description}
  \input{classes/AsistenciaSesionParlamentaria.tex}
\item[{\sf Vocabularios:}] \hfill \\
  Las asistencias a sesiones parlamentarias posiblemente son un
  concepto demasiado local.
\item[{\sf Recomendación para URI:}] \hfill \\
  Dado que los parámetros que definen una asistencia a una sesión son
  la sesión y el parlamentario, se recomienda usar URIs definidas en
  función de estos. Por ejemplo: \\
  {\scriptsize\url{http://data.bcn.cl/id/AsistenciaSesionParlamentaria/SesionParlamentaria/<id-sesion>/Parlamentario/<id-parlamentario>}}
\end{description}

\item[{\sf Vocabularios:}] \hfill \\
  Las asistencias a sesiones parlamentarias posiblemente son un
  concepto demasiado local.
\item[{\sf Recomendación para URI:}] \hfill \\
  Dado que los parámetros que definen una asistencia a una sesión son
  la sesión y el parlamentario, se recomienda usar URIs definidas en
  función de estos. Por ejemplo: \\
  {\scriptsize\url{http://data.bcn.cl/id/AsistenciaSesionParlamentaria/SesionParlamentaria/<id-sesion>/Parlamentario/<id-parlamentario>}}
\end{description}

\item[{\sf Vocabularios:}] \hfill \\
  Las asistencias a sesiones parlamentarias posiblemente son un
  concepto demasiado local.
\item[{\sf Recomendación para URI:}] \hfill \\
  Dado que los parámetros que definen una asistencia a una sesión son
  la sesión y el parlamentario, se recomienda usar URIs definidas en
  función de estos. Por ejemplo: \\
  {\scriptsize\url{http://data.bcn.cl/id/AsistenciaSesionParlamentaria/SesionParlamentaria/<id-sesion>/Parlamentario/<id-parlamentario>}}
\end{description}

\item[{\sf Vocabularios:}] \hfill \\
  Las asistencias a sesiones parlamentarias posiblemente son un
  concepto demasiado local.
\item[{\sf Recomendación para URI:}] \hfill \\
  Dado que los parámetros que definen una asistencia a una sesión son
  la sesión y el parlamentario, se recomienda usar URIs definidas en
  función de estos. Por ejemplo: \\
  {\scriptsize\url{http://data.bcn.cl/id/AsistenciaSesionParlamentaria/SesionParlamentaria/<id-sesion>/Parlamentario/<id-parlamentario>}}
\end{description}
