\subsubsection{\#PersonaNatural}

Las personas naturales son definidas por atributos como los nombres,
los apellidos materno y paterno, la fecha y lugar de nacimiento, el
género (hombre o mujer), publicaciones, estudios, y otra gran cantidad
de atributos que se irán agregando progresivamente.

\begin{description}
  \input{node_classes/class_PersonaNatural.tex}
\item[{\sf Vocabularios relacionados:}] foaf:Person. La mayoría de los
  atributos salientes también encuentran sus pares en los predicados
  de FOAF.
\item[{\sf Recomendación para URI:}] En Chile las personas son
  identificadas por el Rol Único Nacional (RUN), no obstante, este
  suele es considerado información confidencial. Por ende, en la
  publicación se recomienda utilizar un registro de personas
  utilizando un identificador paramétrico. Por ejemplo:

  \url{http://datos.bcn.cl/id/persona/123}

  En este caso el número 123 identificaría a una persona. La práctica
  de usar los nombres de las personas como identificadores de ellas se
  descarta debido a la alta probabilidad de que dos personas poseean
  el mismo nombre.
\end{description}
