\subsubsection{\#ProyectoDeAcuerdoCámaraDiputados}

Proyecto de Acuerdo es la proposición que cinco o más Diputados
presentan por escrito a la Sala, con el objeto de adoptar acuerdos o
sugerir observaciones sobre los actos del Gobierno, o bien, obtener un
pronunciamiento de la Corporación sobre temas de interés general,
tanto nacionales como internacionales, que expresen la preocupación
por ello, de la Cámara.

Por ejemplo, en la sesión 101, del 19 de Agosto de 2010, de la
legislatura 358 hubo dos proyectos de acuerdo:
el número 159, ``Pensión de invalidez para quienes sufren leucemia'' y
el número 158, ``Solicita caducar la concesión de servicio público de
distribución de energía eléctrica a la empresa CGE y su filial
Conafe''. La @resolución indica lo que se resuelve en la sesión sobre
los acuerdos, en este caso, 159 fue aprobado y el 158 fue
rechazado. El atributo @sesiónIngreso indica la sesión en la que los
proyectos se ingresan, que no es comunmente la misma de la resolución,
de éstos. En el ejemplo, ambos proyectos fueron ingresados en la
sesión 66 del 19 de Agosto, el primero a las 12:02 horas y el segundo
a las 11:37.

\begin{description}
  \input{class_ProyectoDeAcuerdoCamaraDiputados.tex}
\end{description}