\subsubsection{\#Pacto}

Los pactos son asociaciones que se hacen entre los partidos para las
elecciones. Es decir, los pactos sólo son válidos para unas
\#Elecciones en particular.

Los tres tipos posibles son: pacto-partido, pacto-independiente,
independiente-independiente. No existe independiente-partido.

Cuando un individuo no tiene pacto, quizá no debiera agregarse el
pacto independiente. El problema es que en la semántica de RDF no
tenemos la opción de decir, ``no tiene pacto''. Lo mismo cuando es
independiente en el partido, luego hay que darle unas vueltas más a
este tema.

\begin{description}
  \input{class_Pacto.tex}
\item[{\sf Recomendación para URI:}] La URI de un pacto podría componerse de
  los parámetros que identifican a las \#Elecciones más el @nombre del
  pacto (que debiera ser único dentro de una elección). Así, se
  recomienda concatenar la URI de las \#Elecciones con el segmento
  ``/pacto/'' seguido del nombre del pacto\footnote{
    Hay que discutir como llevar un nombre con caracteres no ASCCI o
    espacios a un nombre incorporable en una URI.
  }. Es decir, se recomienda una estructura de URI como:

  \url{http://datos.bcn.cl/<ruta-elecciones>/pacto/<nombre-pacto>}
\end{description}